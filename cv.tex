\documentclass{tccv}
\usepackage[english]{babel}

\begin{document}

\part{Trevor Sherrard}
\section{Contact Me}
\begin{eventlist}
\personal
    [www.trevorsherrard.com]
    {github.com/sherrardtr4129}
    {+1 (440) 799-2705}
    {tws4129@rit.edu}
    
\end{eventlist}

\section{Summary}
I am a fourth year electrical engineering student at Rochester Institute of Technology. I am currently seeking a Co-Op for Summer 2019.

\section{Work experience}

\begin{eventlist}

\item{January 2018 -- February 2020}
     {R.I.T}
     {ROS Software Architect}

Responsible for creating a distributed software architecture using robot operating system for a multi-agent intelligent material handling system grant project. Participated in gated reviews of implemented software.

\item{May 2018 -- August 2018}
     {Calvary Robotics}
     {Controls Software Co-Op}

Architected, implemented, and tested an embedded, OPC UA based, industrial internet of things (IIoT) performance tracking software platform for industrial manufacturing machines. Aided in the development of Keyence and Cognex based machine vision applications.

\item{May 2017 -- December 2017}
     {D3 Engineering}
     {Embedded Software Co-Op}
     
Developed board support software and various device drivers for multicore embedded advanced driver assistance systems. Prototyped various image processing pipelines using OpenCV. Designed and performed various tests to verify RTOS software functionality.

\item{January 2017 -- May 2017}
     {Alstom Signaling}
     {Train Signaling Engineering Co-Op}

Responsible for writing installation and cut-over plans based off of electrical schematics for train control rooms.

\end{eventlist}

\section{Education}

\begin{yearlist}

\item[B.S Electrical Engineering]{2015 -- 2020}
     {Rochester Institute of Technology}
     
\end{yearlist}


\section{Robotics Projects}

\begin{yearlist}

\item{2018}
     {Kudos (\href{http://bit.ly/Kudos2018}{http://bit.ly/Kudos2018})}
     {A differential drive robot making use of a distributed ROS architecture and an exploratory SLAM algorithm to map out unknown spaces.}

\item{2016}
     {ToolID (\href{http://bit.ly/ToolID2016}{http://bit.ly/ToolID2016})}
     {Automatic tool identification for the computer science house woodshop. }
     
\end{yearlist}


\section{Computer Vision Projects}

\begin{yearlist}

\item{2018}
     {RIT SPEX HAB Horizon Detection (\href{http://bit.ly/RITHAB2018}
     {http://bit.ly/RITHAB2018})}
     {A CLI application using OpenCV to detect the earth's horizon in images taken from a high altitude balloon. This code ran on a Raspberry Pi at 60,000+ feet.}
    
\item{2015}
     {CSH Augmented Reality Logo (\href{http://bit.ly/ARCSH}
     {http://bit.ly/ARCSH})}
     {An Augmented Reality project for the Computer Science House at RIT.}

\end{yearlist}


\section{Technical Skills}
\begin{factlist}

\item{Advanced  level}
     {C, C++, ROS, OpenCV, Python, Cognex and Keyence Systems}

\item{Intermediate level}
     {SLAM, LiDAR, OPC UA, Git, PLC Programming, Electronics Debug Equipment}

\item{Basic level}
     {RTOS, \LaTeX, Verilog, mmWave Radar, Kuka, IIOT, Imaging Science, MATLAB}

\end{factlist}

\section{Leadership}
\begin{yearlist}

\item{2018}
     {Computer Science House E-board}
     {Served as the CSH House History director for the 2018-2019 academic year.}

\end{yearlist}

\end{document}
